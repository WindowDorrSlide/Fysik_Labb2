\documentclass[11p, titlepage, oneside, a4paper]{article}
% Packages
\usepackage{amsmath}
\usepackage{graphicx}
\usepackage{hyperref}
\usepackage[english,swedish]{babel}
\usepackage[
    backend=biber,
    style=authoryear-ibid,
    sorting=ynt
]{biblatex}
\usepackage[utf8]{inputenc}
\usepackage[T1]{fontenc}
%Källor
\addbibresource{mall.bib}
\graphicspath{ {./images/} }

% Ändra de rader som behöver ändras
\def\inst{Teknikprogrammet}
\def\typeofdoc{Laborationsrapport}
\def\course{Fysik 1 150p}
\def\pretitle{Laboration 2}
\def\title{Krafter: Fjäderkrafter och Normalkrafter}
\def\name{Magnus Silverdal}
\def\username{magnus.silverdal}
\def\email{\username{}@ga.ntig.se}
\def\graders{Magnus Silverdal}

\begin{document}

\begin{titlepage}
		\thispagestyle{empty}
		\begin{large}
			\begin{tabular}{@{}p{\textwidth}@{}}
				\textbf{NTI gymnasiet \hfill \today} \\
				\textbf{\inst} \\
				\textbf{\typeofdoc} \\
			\end{tabular}
		\end{large}
		\vspace{10mm}
		\begin{center}
			\LARGE{\pretitle} \\
			\huge{\textbf{\course}}\\
			\vspace{10mm}
			\LARGE{\title} \\
			\vspace{15mm}
			\begin{large}
				\begin{tabular}{ll}
					\textbf{Namn} & \name \\
					\textbf{E-mail} & \texttt{\email} \\
				\end{tabular}
			\end{large}
			\vfill
            \includegraphics[width=0.5\textwidth]{images/NTI Gymnasiet_Symbol_print_svart.png}
			\vfill
            \large{\textbf{Handledare}}\\
			\mbox{\large{\graders}}
		\end{center}
	\end{titlepage}

    \begin{otherlanguage}{english}
	\begin{abstract}
        The goal with the first experiment was to find the Normal-force and the resistance on a block sliding down a
        wooden slope.
        The second experiment went out for finding the liaison between the expansion of a string with different
        forces.
    \end{abstract}
    \end{otherlanguage}
    % Om arbetet är långt har det en innehållsförteckning, annars kan den utelämnas
	\pagenumbering{roman}
	\tableofcontents
	
	% och lägger in en sidbrytning
	\newpage

	\pagenumbering{arabic}
	
	% i Sverige har vi normalt inget indrag vid nytt stycke
	\setlength{\parindent}{0pt}
	% men däremot lite mellanrum
	\setlength{\parskip}{10pt}
	
	\section{Syfte och frågeställning}
		Syftet med den här experimentet är att hitta samband mellan normalkraften och friktionskraften på en
        liggande klots, på en träbräda.
        Samt att hitta samband mellan en fjäders längd och vikten hängande på fjädern.

	\section{Bakgrund och teori}
        Datan av fjäderns längd samt vikten hängandes på den impoterades i Excel. Sedan beräknades kraften i Newton
        som verkar på den.
        Detta beräknades med formeln $F = m \times g$ vars m är massan i kilogram och g är gravitationskraften.
	

	\section{Metod och materiel}
        \begin{enumerate}
            \item Vagn
            \item Lutande plan med ställning
            \item Linjal
            \item Mobilkamera
        \end{enumerate}
        
        Det lutande planet monteras på ställningen så att den ena änden är 1 dm över bordet, se figur \ref{fig:lutandeplan}. Linjalen placeras längs planet så att det finns en längdskala  i filmen. Kameran placeras vid sidan av uppställningen på ett avstånd så att hela rörelsen ryms i filmen utan att kameran behöver flyttas. Vagnen rullas nedför planet samtidigt som kameran filmar rörelsen. Försöket placeras så att ljusförhållanden och bakgrund ger en så tydlig och skarp film som möjligt.
        
        \begin{figure}[!h]
            \includegraphics[width=0.8\textwidth]{images/lutandePlan.jpg}
            \caption{En blid hade varit superbra här}
            \label{fig:lutandeplan}
        \end{figure}
        
        Filmen analyserades sedan med mjukvaran Tracker för att få fram en tabell med positionen som funktion av tiden.
    \newpage
	\section{Analys och beräkning}
        Datat från analysen av filmen visas i tabell \ref{table:result}
    
        
        \begin{table}
            \begin{center}
            \begin{tabular}{ |c|c| } 
                \hline
                Position (m) & Tid (s)  \\ 
                \hline
                0 & 0  \\ 
                0.1 & 0.02 \\
                \vdots & \vdots \\
                \hline
            \end{tabular}
                \caption{Mätvärden}
                \label{table:result}
            \end{center}
        \end{table}            
        

    Datat importeras i Excel och hastigheten beräknas med hjälp av formeln
    \begin{equation}
        v_m = \frac{\Delta s}{\Delta t}
    \end{equation}
    
    \section{Slutsats och resultat} 
        Resultatet av beräkningarna illustreras i graferna 2 och 3
    \section{Diskussion} 
    Resultatet är perfekt...


    vad är k?
    
    \printbibliography

\end{document}

